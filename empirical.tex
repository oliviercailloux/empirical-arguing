\RequirePackage[l2tabu, orthodox]{nag}
\documentclass[version=3.21, pagesize, twoside=off, bibliography=totoc, DIV=calc, fontsize=12pt, a4paper]{scrartcl}
\input{preamble/packages}
\input{preamble/redac}
\input{preamble/math_basics}
%AF
\NewDocumentCommand{\dy}{}{\bm{+}}%decision yes
\NewDocumentCommand{\dn}{}{\bm{−}}
\NewDocumentCommand{\allformulas}{}{\mathcal{F}}

%Decision Theory (MCDA and SC)
\NewDocumentCommand{\allalts}{}{\mathscr{A}}
\NewDocumentCommand{\allcrits}{}{\mathscr{C}}
\NewDocumentCommand{\alts}{}{A}
\NewDocumentCommand{\dm}{}{i}
\NewDocumentCommand{\allF}{}{\mathscr{F}}
\NewDocumentCommand{\allvoters}{}{\mathscr{N}}
\NewDocumentCommand{\voters}{}{N}
\NewDocumentCommand{\prof}{}{\boldsymbol{P}}
\NewDocumentCommand{\linors}{}{\mathscr{L}(\allalts)}
%Thanks to https://tex.stackexchange.com/q/154549
	%\makeatletter
	%\def\@myRgood@#1#2{\mathrel{R^X_{#2}}}
	%\def\myRgood{\@ifnextchar_{\@myRgood@}{\mathrel{R^X}}}
	%\makeatother
\NewDocumentCommand{\pref}{}{\succ}
\NewDocumentCommand{\prefi}{O{i}}{\succ_{#1}}

%Deliberated Judgment
\NewDocumentCommand{\allargs}{}{\bm{\mathcal{A}}}
\NewDocumentCommand{\args}{}{S}
\NewDocumentCommand{\ar}{}{s}
\NewDocumentCommand{\allprops}{}{T}
\NewDocumentCommand{\prop}{}{t}
\NewDocumentCommand{\ileadsto}{}{⇝}
\NewDocumentCommand{\ibeatse}{}{⊳_\exists}
\NewDocumentCommand{\nibeatse}{}{⋫_\exists}
\NewDocumentCommand{\ibeatsst}{}{⊳_\forall}
\NewDocumentCommand{\nibeatsst}{}{⋫_\forall}
\NewDocumentCommand{\mleadsto}{O{\eta}}{⇝_{#1}}
\NewDocumentCommand{\mbeats}{O{\eta}}{⊳_{#1}}
\NewDocumentCommand{\ibeatseinv}{}{⊳_\exists^{-1}}

%Logic
\NewDocumentCommand{\ltru}{}{\texttt{T}}
\NewDocumentCommand{\lfal}{}{\texttt{F}}

% Defeasible Logic (Comment from Pierre: I'm not quite sure how to use NewDocumentCommand yet)
\newcommand{\curly}{\mathrel{\leadsto}}
\newcommand{\overbar}[1]{\mkern 1.5mu\overline{\mkern-1.5mu#1\mkern-1.5mu}\mkern 1.5mu}

%I find these settings useful in draft mode. Should be removed for final versions.
	%Which line breaks are chosen: accept worse lines, therefore reducing risk of overfull lines. Default = 200.
		\tolerance=2000
	%Accept overfull hbox up to...
		\hfuzz=2cm
	%Reduces verbosity about the bad line breaks.
		\hbadness 5000
	%Reduces verbosity about the underful vboxes.
		\vbadness=1300

\title{Study argument decisiveness empirically}
\author{Olivier Cailloux}
\affil{Université Paris-Dauphine, PSL Research University, CNRS, LAMSADE, 75016 PARIS, FRANCE\\
	\href{mailto:olivier.cailloux@dauphine.fr}{olivier.cailloux@dauphine.fr}
}
\author{And more!}
\affil{Affil2}
\hypersetup{
	pdfsubject={},
	pdfkeywords={},
}

\begin{document}
\maketitle

\section{Rough sketch}
\begin{enumerate}
	\item It is clear that arguments (?) that are not attacked are of prominent importance in many (otherwise different) perspectives about argumentation. Note that “not attacked” depends on which set of arguments we look into; we consider for this discussion that the set of arguments encompasses every arguments that are possibly relevant to the topic at hand (and will come back to this). In this context, we call decisive the non-attacked arguments.
	\item sometimes considered known a priori: logical representation; or simply given
	\item reasons not to consider it known: what counts is sometimes (at least in part) what the receiver understands, not what logic says; there may be implicit information not encoded in the logic representation; we may not have expertise to encode the arguments logically; encoding may be suspected to be non neutral wrt the debated issue
	\item what if not known a priori? we consider that what counts is what the receiver of the argument says and we propose to study several aspects related to this perspective: how to find decisive arguments in this sense; is this consensual; what are appropriate methods, statistical or others; what are the protocols that must be used to gather arguments; how can we test that every arguments that are possibly relevant have been considered?
	\item a sketch of a possible approach: let people come to a website and argue and indicate which arguments theirs answer, then comfront this to the opinion of other visitors
	\item possibly, contrast with persuasion?
	\item link to explainable AI (should focus on decisive arguments), exploitation of weaknesses of will or sub-conscious behavioral patterns (youtube autoplay; nudge; anchoring) VS reflective arguments (advocacy), and the like? Link to deliberative democracy?
	\item review existing experimental work in FAT? Other related works? Philosophical approaches (Rawls, Habermas)?
\end{enumerate}

\section{Introduction}
\label{sec:intro}

%\bibliography{bibl}

\end{document}

